% Source: https://zenkelab.org/resources/latex-rebuttal-response-to-reviewers-template/
% Modified with minor justifications by Hongyuan Jia
%
% Original License:
%
% LaTeX rebuttal letter example.
%
% Copyright 2019 Friedemann Zenke, fzenke.net
%
% Based on examples by Dirk Eddelbuettel, Fran and others from
% https://tex.stackexchange.com/questions/2317/latex-style-or-macro-for-detailed-response-to-referee-report
%
% Licensed under cc by-sa 3.0 with attribution required.
% See https://creativecommons.org/licenses/by-sa/3.0/
% and https://stackoverflow.blog/2009/06/25/attribution-required/

\documentclass[11pt]{article}
\usepackage[utf8]{inputenc}
\usepackage{fullpage}
\usepackage{xcolor}

\usepackage[hidelinks]{hyperref}

% import Eq and Section references from the main manuscript where needed
% \usepackage{xr}
% \externaldocument{../paper}

% package needed for optional arguments
\usepackage{xifthen}
% define counters for reviewers and their points
\newcounter{reviewer}
\setcounter{reviewer}{0}
\newcounter{point}[reviewer]
\setcounter{point}{0}

% This refines the format of how the reviewer/point reference will appear.
\renewcommand{\thepoint}{P\,\thereviewer.\arabic{point}}

% command declarations for reviewer points and our responses
\newcommand{\reviewersection}{\stepcounter{reviewer} \bigskip \hrule
                  \section*{Reviewer \thereviewer}}

\newenvironment{point}
   {\refstepcounter{point} \bigskip \noindent {\textbf{Reviewer~Point~\thepoint} } ---\ }
   {\par }

\newcommand{\shortpoint}[1]{\refstepcounter{point}  \bigskip \noindent
    {\textbf{Reviewer~Point~\thepoint} } ---~#1\par }

\newenvironment{reply}
   {\medskip \noindent \textbf{Reply}:\  }
   {\medskip }

\newcommand{\shortreply}[2][]{\medskip \noindent \begin{sf}\textbf{Reply}:\  #2
    \ifthenelse{\equal{#1}{}}{}{ \hfill \footnotesize (#1)}%
    \medskip \end{sf}}

% change the font color in quote
\AddToHook{env/quote/begin}{\small\color{blue!70}}

% for pandoc generated list
\providecommand{\tightlist}{\setlength{\itemsep}{0pt}\setlength{\parskip}{0pt}}

\begin{document}
\hypertarget{response-to-the-reviewers}{%
\section*{Response to the reviewers}\label{response-to-the-reviewers}}
\addcontentsline{toc}{section}{Response to the reviewers}

Dear Reviewers,

Thank you for giving me the opportunity to submit a revised draft of my
manuscript titled \emph{Epwshiftr: incorporating open data of climate change
prediction into building performance simulation for future adaptation and
mitigation}. We appreciate the time and effort that you have dedicated to
providing valuable feedback and insightful comments on our work. We have been
able to incorporate changes to reflect most of the suggestions and have
highlighted the changes within the uploaded manuscript.

Below is a point-by-point response to the reviewers' comments and concerns.

\reviewersection

\begin{point}
This paper developed a free open-source future weather generator epwshiftr,
which can generate future weather data for 11 meteorological variables.

\end{point}

\begin{reply}
Thank you for the comments. We hope epwshiftr can ease the data-processing
burden and reduce the costs of energy modelers when preparing future weather
files for assessing climate change impacts on building energy performance.

\end{reply}

\reviewersection

Epwshiftr appears to be a useful tool for modelers investigating likely
performance of buildings in the future.

\begin{point}
The format and writing quality of the paper is acceptable once minor errors are
corrected. I noticed as least one incorrect word usage (``further'' instead of
``future''). Please carefully proof-read the paper.

\end{point}

\begin{reply}
Thank you for pointing this out. The typo has been corrected. The manuscript has
been proofread with spelling and grammatical errors corrected.

\end{reply}

\begin{point}
Figures 1 and 2 are both unreadable in a printout of the paper. They are not
particularly clear even when viewing the PDF at high magnification. The listings
are fuzzy; listings 4 and 5 are also too small. The manuscript is significantly
shorter than the 8 page limit, so space is available to enlarge and clarify
these items.

\end{point}

\begin{reply}
Thank you for the suggestions.

The manuscript has been rewritten using BS2023 \LaTeX template for better image
and listing formatting.
The image file of Figure 1 and 2 have been replaced with high-resolution ones.
Also, both have been enlarged and placed in full page width.

In the original draft, all listings were screenshots of code snippets in small sizes.
Now we replace all of them by rewriting using \LaTeX listings. Clarifications of
code have been added as comments in each listing. The fonts have been enlarged
to make sure they are readable in the printout version.

\end{reply}

\begin{point}
You refer to the morphing method as ``reliable''. I am not sure there is
consensus on that. Morphing modifies individual weather items independently and
may not preserve physically realistic relationships among them. Further,
unconstrained morphing can produce physically impossible values, such as wet
bulb temperature greater than dry bulb temperature. Finally, morphing does not
capture time-dependent effects such as concentration of temperature increases in
nighttime hours. You should amplify your discussion of how morphing is
implemented.

\end{point}

\begin{reply}

\label{rp:validation}
Thank you for the insightful comments and suggestions.

We have changed the description of the morphing method in both the Abstract and
Introduction section. Both advantages and disadvantages of it have been added in
the Introduction section:

\begin{quote}
Morphing can capture the average future weather conditions from GCM while
preserving historical weather sequences. It requires low computational power,
making it possible to create many weather files from worldwide locations.
However, morphing may under- or overestimate climate change impacts because of
the lacking of ability to capture future extreme weather conditions and
potential differences in the reference time frame of the TMY and GCM data
(Moazami et al., 2019). Moreover, careful consideration should be given to
morphing when modifying individual meteorological variables independently,
breaking their physical relationships. Despite these shortcomings, this method
is still widely used because of its simple and flexible characteristics.
\end{quote}

Description about how the morphing method is implemented in epwshiftr has also
been added:

\begin{quote}
To avoid unrealistic results, the \emph{Morphing Module} has taken extra data
validation and calculation steps, including but not limited to:

\begin{itemize}
\tightlist
\item
  Warnings are generated if there are any missing values in the input EPW and
  GCM data.
\item
  Unit conversions between data of EPW and GCM are automatically performed using
  the units (Pebesma et al., 2016) R package, e.g.~all temperature data have
  been converted to Celsius before calculation.
\item
  Calculation of dew point temperature is performed based on dry-bulb
  temperature and relative humidity using the psychrolib (Meyer and Thevenard,

  \begin{enumerate}
  \def\labelenumi{\arabic{enumi})}
  \setcounter{enumi}{2018}
  \tightlist
  \item
    R package.
  \end{enumerate}
\item
  Input values of relative humidity that exceed 100\% will be reset to 100\%.
\item
  A threshold value is set for the stretch factor (\(\alpha\)), i.e.~monthly-mean
  fractional change, when performing morphing operations. The default value is
  set to \texttt{3}. If the absolute \(\alpha\) exceeds this threshold value, warnings
  are issued to suggest users further investigate the input data before
  continuing. Moreover, the morphing method will use the shift factor (\(\Delta x\))
  to to avoid unrealistic morphed values.
\end{itemize}
\end{quote}

\end{reply}

\begin{point}
Further, you should comment on the adaptability of your framework to support
alternative extrapolation algorithms.

\end{point}

\begin{reply}

Thank for the suggestion.

Currently, epwshiftr only supports the morphing method. We are happy to explore
the feasibility of supporting alternative extrapolation algorithms in the
future. Clarifications have also been added in the \emph{Morphing Module} section:

\begin{quote}
Currently, epwshiftr only supports the morphing method. But the \texttt{morphing\_epw}
interface provides parameters to modify which factors should be used for each
meteorological variable, with meaningful defaults value given. For example,
radiation-related variables are, by default, morphed using the stretch factor,
avoiding unrealistic positive values at nighttime. The modular design pattern of
epwshiftr makes it decouple the data structure and the actual extrapolation
algorithm used. We are happy to explore the feasibility of supporting
alternative extrapolation algorithms in the future.
\end{quote}

\end{reply}

\begin{point}
In addition, it would be helpful to provide at least general information on
testing and validation. Have simulation results been compared for studies done
with epwshiftr weather data to those conducted with some of the other future
weather methods you cite? Is the NetCDF data reliable and what happens when it
is not? Can potential users be confident that epwshiftr will ``do the job''?

\end{point}

\begin{reply}

Thank you for the comments and suggestions.

We take seriously about code correctness. For details about the test coverage,
please see epwshiftr code coverage
report\footnote{\url{https://app.codecov.io/gh/ideas-lab-nus/epwshiftr?branch=master}}. A
description of the code quality assurance has been added in the 3rd section:

\begin{quote}
The epwshiftr package follows the Test-Driven Development (TDD) process. Around
450 unit tests are carefully made, covering 94\% of the codebase. They are
automatically run on Windows, macOS, and Linux whenever changes are made in
epwshiftr on CRAN and GitHub.
\end{quote}

In terms of validation, besides data validation steps in epwshiftr itself
described in Response \ref{rp:validation}, epwshiftr uses Defensive Programming
pattern\footnote{\url{https://en.wikipedia.org/wiki/Defensive_programming}}. Careful efforts
have been made to validate user inputs including data types, value length and
range, ESGF data node availability, NetCDF file accessibility, etc., avoiding
common mistakes before computation.

epwshiftr stores and returns data in a consistent tidy format, making it easy
for users to examine the returned data of each module further. In terms of the
calculated morphing factors, a description of the structure of the returned data
from the \emph{Morphing Module} is also added:

\begin{quote}
Besides the efforts above, the \emph{Morphing Module} always returns the calculated
\(\Delta x\) and \(\alpha\) values in dedicated columns, which provides
opportunities for detailed examination and custom statistical analyses.
\end{quote}

\end{reply}
\end{document}
